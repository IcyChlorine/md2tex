\documentclass[UTF8]{ctexart}
\usepackage{geometry}
\geometry{a4paper,centering,scale=0.8}

\usepackage{float}
\usepackage{graphicx}
\usepackage{listings}

%程序框设置
\usepackage{color,xcolor}
\usepackage{fontspec}
\setmonofont{Consolas}
\definecolor{mygreen}{rgb}{0.4,0.4,0.4}
\definecolor{mygray}{rgb}{0.96,0.96,0.96}
\definecolor{mymauve}{rgb}{0.58,0,0.82}
\definecolor{dark_blue}{rgb}{0.15,0.05,0.55}
\lstset{
	%编程语言
	language=python,
	%代码框的基本样式
	frame=single,                    % 设置有边框
	numbers=left,                    % 设置行号
	%字体设置
	numberstyle=\footnotesize\sffamily,% 设置行号字体(及字号)
	basicstyle=\ttfamily,            % 代码正文字体
	%颜色设置
	backgroundcolor=\color{mygray},  % 代码框背景颜色
	rulesepcolor=\color{black},      % 设置边框颜色
	commentstyle=\color{mygreen},    % 设置注释颜色
	keywordstyle=\color{blue},       % 设置keyword颜色
	identifierstyle=\color{dark_blue},% 设置标识符颜色
	stringstyle=\color{violet},      % 设置字符串颜色
	%一些排版细节
	columns=fixed,                   % 字间距固定
	breaklines=true,                 % automatic line breaking only at whitespace
	captionpos=b,                    % sets the caption-position to bottom
	tabsize=4,                       % 把tab扩展为4个空格,默认8个太长了
	%特殊设置
	escapeinside={\%*}{*)}           % if you want to add LaTeX within your code
}


\title{md2tex}
\author{IcyChlorine\footnote{icy\_chlorine@pku.edu.cn}}
\date{\today}

\begin{document}
	\maketitle


\begin{figure}[H]
	\centering
	\includegraphics[width=1.\textwidth]{md2tex.png}
	\caption{}
\end{figure}


\textbf{md2tex} is a python tool that converts markdown source codes into latex source, while preserving most of the important formats such as titles, sections and formulas.

md2tex is in its \textbf{beta ver.} currently.

\section{Initiative}

\textbf{Latex} is the standard and most powerful typesetting software within the domain of science and technology, but it is relatively difficult to write latex codes. On the other hand, \textbf{markdown} is a lightweight mark language that is easier to edit. Thanks to `typora`, markdown can be rendered in real-time and make editing much easier. Lightweight as it is though, markdown still supports most elements required for writing an elegant article or technical note, including subtitle, formula and code environment, enumerate environment. 

Realizing that latex and markdown share a lot of similar features and functionalities, I started to ponder whether I can write a markdown to latex convertor, so that I can enjoy the merits of both latex and markdown. That's why I started working on this project. Now md2tex can convert most of the useful markdown features into latex, and it has turned out to be quite useful for me.

\section{How to use}

Python3 environment is required. `md2tex` is simple to use: 

\begin{lstlisting}
py md2tex.py <input\_filename> [-o <output\_filename>]
\end{lstlisting}

The program reads in a markdown source file (`\emph{.md`) and outputs a latex source file(`}.tex`). Note that the output file is still a latex source file and further compilation is needed to get a pdf file. 

\section{Some Notes}

I personally use `vscode` to further edit and compile latex source into final files. My `vscode` settings for latex is in `./vscode\_sample\_for\_latex/` directory as reference.

There're some demo cases that you can refer to in `./demo/`. You can also try to convert this `README.md` file into latex to see what will happen!

Note that `md2tex` doesn't guarantee that the generated latex code is compliable. It only converts the key elements and there maybe subtle bugs(usually rare) that have to be handled manually.

`md2tex` is built to be \textbf{customizable}.  You can easily customize the latex source generated by modifying `sym2tex\_template.json`. For example, you can change the preamble to be generated by doing so.

\begin{center}
	*\quad*\quad*
\end{center}

\section{What it can do}

Listed below are formats that `md2tex` can convert now:

\begin{itemize}	\item Title
	\item Section
	\item \emph{Italic} and \textbf{Bold} texts
	\item Formula
	\item Code (multi-line only)
	\item Figure
	\item "$\backslash$newcommand" control sequence in formula
	\item Delim lines
	\item Enumerations(such as the one here)
	\item Quotation(recognizable but can't transform)
\end{itemize}
See `./demo/` to get see some of the formats above converted into latex sources.

\section{What it can not do}

There're also some known formats that `md2tex` can't convert/can't handle properly for the time being:

\begin{itemize}	\item Nested Enumerations.
	\item Inline code environments.(they will be left unchanged, and there's plan to support it in future versions)
	\item lines begin with "\#\#\# " in code environment will be falsefully recognized as section.
	\item Special math environment , such as `$\backslash$beign\{align\}` in multiline formulas.
	\item italic and bold environments that interleaves with each other(even in typora this cannot be correctly recognized, actually).
\end{itemize}
\section{Some additional notes}

\textbf{Q}: How it works?

\textbf{A}: `md2tex` first parse markdown source files into an intermediate representation, that convert the intermediate representation to latex source files. Modify `DEBUG\_PRINT` as `True` in `constants.py` to see the intermediate representations.

\textbf{Q}: It's widely known that typora has a built-in link for markdown-to-latex convertor, which leads to a tool called `pandoc`. What's the difference between `md2tex` and `pandoc`?

\textbf{A}:  `pandoc` can convert markdown source into latex as well, but there's a few differences between this project and`pandoc`.

`pandoc` try to convert all details and formats from markdown to latex, but this will make the generated latex obscure and sometimes hard to compile. On the other hand, `md2tex` only try to convert the most elementary elements into latex, so the code generated will be clearer, simpler and easier to read or compile. 

Moreover, `md2tex` is built to be customizable. You can easily customize the generated latex source by changing `md2sym\_template.json` and `sym2tex\_template.json`.

\end{document}