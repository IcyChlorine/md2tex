\documentclass[UTF8]{ctexart}
\usepackage{geometry}
\geometry{a4paper,centering,scale=0.8}

\usepackage{amsmath}
\usepackage{amssymb}



\title{Notes: \emph{Probabilistic Cloning and Identification of Linearly Independent Quantum States}}
\author{IcyChlorine\footnote{icy\_chlorine@pku.edu.cn}}
\date{\today}

\newcommand\bra[1]{\left\langle#1\right|}
\newcommand\ket[1]{\left|#1\right\rangle}
\newcommand\state[1]{\ket{#1}}
\newcommand\qinner[2]{\left\langle#1\right.\left|#2\right\rangle}
\newcommand\qouter[2]{\left|#1\right\rangle\left\langle#2\right|}
\renewcommand\a{\alpha}
\renewcommand\b{\beta}
\newcommand\g{\gamma}
\newcommand\diag{\mathrm{diag}}
\newcommand\abn[1]{#1_1,#1_2,\cdots,#1_n}\begin{document}
	\maketitle


\begin{center}
	*\quad*\quad*
\end{center}

论文作者:段路明老师,郭光灿院士
Notes by IcyChlorine 2021-8-18

\begin{center}
	*\quad*\quad*
\end{center}

\section{Overview}

一般的幺正操作不能克隆未知的量子态,除非量子态只能从一组正交的态中挑选

但这不代表加上测量之后不行!

作为代价的是:

1.不能可靠地克隆,而受限于成功概率$\gamma_i$; 论文作者还给出了$\gamma_i$的理论上界

2.仍然要求量子态从一个已知的态的集合$\{\ket{\Psi_1},\ket{\Psi_2},\cdots,\ket{\Psi_n}\}$中选取,并且要求这些态线性无关;不过这已经从 {相互正交的态}扩展到了{线性无关的态}

\section{Contents}

\begin{itemize}	\item No-cloning Theorem
	\item 克隆一份量子态:将测量作为投影算子使用
	\item 成功克隆的概率$\g_i$的上界
	\item 克隆多份量子态
	\item 从量子态克隆到量子态分辨
\end{itemize}
\begin{center}
	*\quad*\quad*
\end{center}

\section{Recall: Basic No-cloning Theorem}

\textbf{No-cloning Theorem: 不能可靠地克隆一个未知的量子态}

准确来说,不存在幺正操作$U$,使得$\ket{\Psi_i}\ket{\Sigma}\xrightarrow{U}\ket{\Psi_i}\ket{\Psi_i},\ \forall\ket{\Psi_i}\in\mathcal{H}$

\section{将测量作为投影算子使用}

考虑幺正算符:
$$
%一些常用命令的预定义
U\big[\ket{\Psi_i}\ket{\Sigma}\ket{P_0}\big]=\sqrt{\g_i}\ket{\Psi_i}\ket{\Psi_i}\ket{P_0}+\sum_jc_{ij}\ket{\Phi_{AB}^{(j)}}\ket{P_j}\label{prob_clone_proc_EASY}
$$
其中$\g_i\in(0,1]\subset\mathbb{R}$,$\ket{\Phi_{AB}^{(j)}}$表示系统AB的不论什么奇奇怪怪的态——我们无所谓。

如果存在这样的算符$U$,在作用到系统AB上之后再对P进行测量,

\begin{itemize}	\item 测量得到$\ket{P_0}$说明量子态克隆成功,保留AB
	\item 测量得到$\ket{P_{j\neq i}}$说明克隆失败,得到了一坨不想要的态,丢弃
\end{itemize}
那么我们就有$\gamma_i$的概率能够成功克隆$\ket{\Psi_i}$.



\textbf{引理2} 形如$X=[\qinner{\psi_i}{\psi_j}]$的矩阵是正定的,当且仅当向量集$\{\ket{\psi_i}\}$是线性无关的;否则,就只是半正定矩阵.

证明 考虑任意向量$\alpha=[\alpha_1,\alpha_2,\cdots,\alpha_n]^T$,则
$$
\alpha^\dagger X\alpha=\sum_{ij}\alpha_i^*\qinner{\psi_i}{\psi_j}a_j\\
=\qinner{\Psi}{\Psi}=||\ket{\Psi}||\ge0
$$
其中$\ket{\Psi}=\sum_i\alpha_i\ket{\psi_i}.$

从而,如果$\ket{\psi_i}$线性相关,$\ket{\Psi}$可以是零向量,上式可以取到等号——矩阵$X$就只能是半正定矩阵;

反之,如果$\ket{\psi_i}$线性无关,上式就永远取不到等号,$X$就是彻底的正定矩阵.



\textbf{证明}:(1)考虑$\langle\eqref{prob_clone_proc_EASY}_i|\eqref{prob_clone_proc_EASY}_j\rangle$拼起来构成的矩阵方程
$$
X^{(1)}=\sqrt\Gamma X^{(2)}\sqrt\Gamma^\dagger+CC^\dagger
$$
其中$X^{(1)}=[\qinner{\Psi_i}{\Psi_j}]$为正定矩阵,$\sqrt\Gamma=\diag\{\sqrt{\g_1},\sqrt{\g_2},\cdots,\sqrt{\g_n}\}=\sqrt\Gamma^\dagger$,$X^{(2)}=\left[\qinner{\Psi_i}{\Psi_j}^2\right]$

\begin{center}
	*\quad*\quad*
\end{center}

\end{document}